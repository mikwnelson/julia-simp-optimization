\documentclass[final]{beamer}
%\usepackage{etex}

\usetheme{boxes}
\usecolortheme{rose}
\usefonttheme{professionalfonts}
%\setbeamertemplate{mini frames}{}
\usepackage[english]{babel}
\usepackage[utf8]{inputenc}
\usepackage[T1]{fontenc}
\usepackage{amsfonts,amssymb,amsmath}
%\usepackage{epstopdf}
\usepackage{algorithmic}
% \usepackage{algorithm}
\usepackage{tikz}
\usetikzlibrary{arrows,shapes}
%\usepackage{xifthen}
\usepackage{pgfplots}
\usepackage{float}
%\usepackage{latexsym}
%\usepackage[dvips]{pstricks} % PSTricks
%\usepackage{pst-node}
%\usepackage{psfrag}
%\usepackage{wrapfig}
%\usepackage{alltt}
%\usepackage{url}
%\usepackage{pict2e}
%\usepackage{multimedia}
%\usepackage{fancyvrb}
\usepackage{graphicx}
%\usepackage{verbatim}
%\usepackage{mathrsfs}
%\usepackage{multirow}
\usepackage{bm}
%\usepackage{algorithm2e}
\usepackage{hyperref}
%\usepackage[angle=0,scale=5,opacity=1,color=black]{background}
%\usepackage[active]{srcltx}
\usepackage{commath}
\usepackage{multicol}
%\usepackage[export]{adjustbox}


%\usetikzlibrary{spy}
%\tikzset{level 1 concept/.append style={font=\sf, sibling angle=60,level distance = 27mm}}
%\tikzset{level 2 concept/.append style={font=\sf, sibling angle=55,level distance = 17mm}}
%\tikzset{every node/.append style={scale=0.6}} 


%% Beamer Style Setup
%\definecolor{mywine}{rgb}{.5412,.0235,.0}
%\usecolortheme[named=mywine]{structure}
%\setbeamercolor{title}{fg=mywine}
%\setbeamercolor{frametitle}{fg=mywine}
%\setbeamercovered{transparent}
\setbeamertemplate{navigation symbols}{}

%% ALGORITHM Commands %%%%%%%%%%%%%%%%%%%%%%%%%%%%%%
\renewcommand{\algorithmicrequire}{\textbf{Input:}}
\renewcommand{\algorithmicensure}{\textbf{Output:}}
\newfloat{program}{thp}{Program}
\floatname{program}{}
\renewcommand*\theprogram{}
\newcommand{\theHalgorithm}{\arabic{algorithm}}
\newcommand{\eq}[1]{(\ref{#1})}




\newtheorem{remark}[theorem]{Remark}
\newtheorem{proposition}[theorem]{Proposition}


\renewcommand{\Im}{{\ensuremath{\mathrm{Im\,}}}}
\renewcommand{\Re}{{\ensuremath{\mathrm{Re\,}}}}

\newcommand{\R}{\mathbb{R}}
\newcommand{\C}{\mathbb{C}}
\newcommand{\N}{\mathbb{N}}



\selectlanguage{english}
\title[SIMP Optimization]{\bf Topological Optimization Using the SIMP Method}
\author[Mikal Nelson]{Mikal Nelson
%(joint work with XXXXX) 
}
\institute[KU]{
{University of Kansas}\\
Department of Mathematics\\
mikal.nelson@ku.edu\\
}
\date[M.A. Thesis Defense]{
\small M.A. Thesis Defense\\
July 26\textsuperscript{th}, 2021\\
Zoom
}
\titlegraphic{\hspace{0in}\includegraphics[scale=0.35]{logos/ku_math_logo.png}\;
\hspace{1.2in}\includegraphics[scale=0.12]{logos/ku_logo}\;

}

\graphicspath{ {pictures/} }

\begin{document}

\frame{
\titlepage
}

\frame{
	
	\structure{\bf Outline}
	
	\medskip
	\begin{itemize}
		\item Orthogonal Polynomials
		\item Chebyshev Polynomials
		\item Chebyshev Expansion
		\item Using Kernel Polynomials
		\item Application: Calculating the Density of States
	\end{itemize}
	%
	
}

\frame{

\structure{\bf Orthogonal Polynomials}

\medskip
Scalar product on $[a,b]$: $$\left\langle f|g \right\rangle = \int_a^b w(x)f(x)g(x)\dif x$$
Given a scalar product, we get a set of polynomials, $p_n$, which satisfy the orthogonality relation $$\left\langle p_n|p_m \right\rangle = \delta_{n,m}\left\langle p_n|p_n \right\rangle.$$
This allows for an expansion of any given function $f(x)$ in terms of the $p_n(x)$:
$$f(x)=\sum_{n=0}^{\infty}\alpha_n p_n(x)\quad\text{with}\quad \alpha_n=\frac{\left\langle p_n|f\right\rangle}{\left\langle p_n|p_n \right\rangle}$$
%

}

\frame[plain]{
	
	\structure{\bf Chebyshev Polynomials}
	
	\medskip
	I will focus on Chebyshev polynomials of the first kind, $T_n$:\\\vspace{0.25cm}
	Defined on interval $[-1,1]$ with weight function $w(x)=\left(\pi\sqrt{1-x^2}\right)^{-1}$.
	$$T_n=\cos\left(n\arccos(x)\right)$$
	Recursively defined:
	$$T_0(x)=1,\quad T_{-1}(x)=T_1(x)=x,$$
	$$T_{m+1}(x)=2xT_m(x)-T_{m-1}(x).$$
	The polynomials also follow the relation
	$$2T_m(x)T_n(x)=T_{m+n}(x)+T_{m-n}(x).$$
	Chebyshev Polynomials have some particular advantages which make them ideal for use in orthogonal polynomial expansions.
	%
	
}

\frame[plain]{
	
	\structure{\bf Chebyshev Expansion}
	
	\medskip
	The expansion of $f(x)$ in terms of Chebyshev polynomials
	$$f(x)=\sum_{n=0}^{\infty}\frac{\left\langle f|T_n\right\rangle_1}{\left\langle T_n|T_n\right\rangle_1}T_n(x)$$
	To make this a easier to compute, we rearrange:
	$$f(x)=\frac{1}{\pi\sqrt{1-x^2}}\left[\mu_0+2\sum_{n=1}^{\infty}\mu_n T_n(x)\right],\quad \mu_n=\int_{-1}^{1}f(x)T_n(x)\dif x.$$
	However, calculating the moments $\mu_n$ can be quite computationally expensive.
	%
	
}

\frame[plain]{
	
	\structure{\bf Motivation for KPM: Gibbs Oscillations}
	
	\medskip
	In practice, we cannot compute an infinite series, so we need to truncate:
	$$f(x)\approx\frac{1}{\pi\sqrt{1-x^2}}\left[\mu_0+2\sum_{n=1}^{N-1}\mu_n T_n(x)\right]$$
	Gibbs Oscillations:
	%
	
}

\frame[plain]{
	
	\structure{\bf Kernel Polynomials}
	
	\medskip
	To lessen the impact of the Gibbs Oscillations, we introduce a ``damping function'' in the form of a kernel polynomial, $g_n$:
	$$f_{\text{KPM}}(x)=\frac{1}{\pi\sqrt{1-x^2}}\left[g_0\mu_0+2\sum_{n=1}^{N-1}g_n\mu_n T_n(x)\right]$$
	%
	
}

\frame[plain]{
	
	\structure{\bf Application: Density of States Calculations}
	
	\medskip
	Density of States: How many energy states exist at a given energy $E$.
	\begin{multicols}{2}
	\begin{minipage}{0.6\textwidth}
		$$\rho(E)=\frac{1}{N}\sum_{k=0}^{N-1}\delta(E-E_k)$$
		Scale $E$ down to $[-1,1]$: $$E=aX+b$$
		We can approximate $\rho(E)$ using KPM:
		$$\rho(E)\approx\frac{1}{\pi\sqrt{1-x^2}}\left[g_0\mu_0+2\sum_{n=1}^{N-1}g_n\mu_n T_n(X)\right]$$
		$$\mu_n\approx\frac{1}{N_r}\sum_{r}\left\langle r|T_n(X)|r\right\rangle$$
	\end{minipage}
	\end{multicols}
	%
	
}

\frame[plain]{
	
	\structure{\bf Disilicon Si\textsubscript{2} Density of States}
	
	\medskip
	%
	
}

\frame[plain]{
	
	\structure{\bf Twisted Graphene Density of States}
	
	\medskip
	%
	
}

\frame[plain]{
	
	\structure{\bf Topics of Further Exploration}
	
	\medskip
	\begin{itemize}
		\item Effects of the number of random vectors ($R$) used in calculating the moments.
		\item Optimal resolution values ($N$) for various Hamiltonians.
		\item More in-depth comparisons of various kernels.
	\end{itemize}
	%
	
}

\frame[plain]{
	
	\structure{\bf Acknowledgments}
	
	\medskip
	Big thanks to Professor Cazeaux for helping me to get my program to work.\\ \vspace{0.5cm}
	Also thanks to Aaron for talking through some aspects of the programming with me.
	%
	
}

\frame[plain,allowframebreaks]{
\structure{\bf Bibliography}

\medskip
\begin{tiny}
\bibliographystyle{plain}
\bibliography{Thesis_Bibliography.bib}
\nocite{Boyd2004}
\end{tiny}
}

\end{document}
 
