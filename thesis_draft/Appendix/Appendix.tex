
\chapter{Julia Codes}

\section{Backtracking Line Search}
Implementation of the Backtracking Line Search in Julia with default values for the parameters being $\alpha=0.25$ and $\beta=0.5$.
\begin{jllisting}
	function ln_srch(d_dir,x,f,fx,dfx;alpha=0.25,beta=0.5)
		t = 1
		x1 = x+t*d_dir
		y1 = f(x1)
		y2 = fx+alpha*t*(dfx)'*d_dir
		while y1 > y2
			t = beta*t
			x1 = x+t*d_dir
			y1 = f(x1)
			y2 = fx+alpha*t*(dfx)'*d_dir
		end
		return t
	end
\end{jllisting}

\section{Gradient Descent}\label{alg:grad-desc}
\begin{jllisting}
	using LinearAlgebra
	
	#Function to Optimize
	f(x)=(x[2])^3-x[2]+(x[1])^2-3x[1]
	
	#Gradient of Function
	df(x)=[2x[1]-3,3x[2]^2-1]
	
	#Initial Point
	x=[0,0]
	
	#Line Search Algorithm
	function ln_srch(d_dir,x,f,fx,dfx;alpha=0.25,beta=0.5)
		t = 1
		x1 = x+t*d_dir
		y1 = f(x1)
		y2 = fx+alpha*t*(dfx)'*d_dir
		while y1 > y2
			t = beta*t
			x1 = x+t*d_dir
			y1 = f(x1)
			y2 = fx+alpha*t*(dfx)'*d_dir
		end
		return t
	end
	
	#Gradient Descent Algorithm
	function grad_d(f,df,x)
		d_dir = -df(x)
		t = ln_srch(d_dir,x,f,f(x),df(x))
		x = x + t*d_dir
		return x
	end
	
	#Compute Minimum for Defined Tolerance
	while norm(df(x))>0.00001
		global x = grad_d(f,df,x)
	end
	
	display(x)

	
\end{jllisting}

\section{Nonlinear Conjugate Gradient}
\begin{jllisting}
using LinearAlgebra

i = 0
k = 0

#Function to Optimize
f(x)=(x[2])^3-x[2]+(x[1])^2-3x[1]

#Gradient of Function
df(x)=[2x[1]-3,3x[2]^2-1]

#Hessian of Function
hf(x)=[2 0; 0 6x[2]]

#Initial Point
x = [0,0]

n = size(x)[1]

r = -df(x)

d = r

delta_new = (r')*r

delta_0 = delta_new

#Choose Max Iterations
i_max = 100

#Choose Max Newton-Raphson Iterations
j_max = 10

#Set CG Error Tolerance
epsilon_CG = 0.5

#Set Newton-Raphson Error Tolerance
epsilon_NR = 0.5

while (i < i_max) && (delta_new > (((epsilon_CG)^2)*(delta_0)))
	global j = 0
	global delta_d = (d')*d
	while true
		global alpha = -((df(x))'*d)/((d')*hf(x)*d)
		global x = x + alpha*d
		global j = j + 1
		(j < j_max) && ((alpha)^2*(delta_d) > (epsilon_NR)) || break
	end
	global r = -df(x)
	global delta_old = delta_new
	global delta_new = (r')*r
	global beta = (delta_new)/(delta_old)
	global d = r + beta*d
	global k = k + 1
	if (k == n) || (((r')*d) <= 0)
		global d = r
		global k = 0
	end
	global i = i + 1
end

display(x)
	
\end{jllisting}

\section{SIMP Method for Volume-to-Point Heat Conduction Problem on 60x60 Control Volume Grid}\label{sec:SIMP-Alg}
\begin{jllisting}
using NLopt, SparseArrays, LinearAlgebra, LaTeXStrings, Plots
pyplot()

##########################
## Fixed Variable Input ##
##########################

p = 1

p_max = 20

p₊ = 0.05

m = 60

n = 60

k₀ = 1.0

k₊ = 100.0

xlen = 0.1

ylen = 0.1

ε₀ = 1e-3 # Outer loop error tolerance

εᵢ = 1e-4 # Inner Loop error tolerance

q = 1e4

##########################
## Compute size of each ##
##   control volume     ##
##########################

Δx = xlen / n
Δy = ylen / m

###########################################
## Create Optimization Problem Structure ##
## Using MMA with dimentions (m+1)*(n+1) ##
###########################################

opt = Opt(:LD_MMA, (m + 1) * (n + 1))

#########################
## Average Temperature ##
## Objective Function  ##
#########################

function av_temp(
	η::Vector,
	grad::Vector,
	p,
	m,
	n,
	xlen = 0.1,
	ylen = 0.1,
	k₀ = 1.0,
	k₊ = 100.0,
)

	#######################
	## Assemble K Matrix ##
	#######################

	##########################
	## Compute size of each ##
	##   control volume     ##
	##########################

	Δx = xlen / n
	Δy = ylen / m

	η = reshape(η, m + 1, n + 1)

	# Define Conductivity Penalization Function for design parameters eta
	k = k₀ .+ (k₊ - k₀) .* η .^ p

	# Control Volumes are designated based on matrix-type coordinates, so that volume [i,j] is the control volume in the i-th row and j-th column from the upper left.

	# Compute conductivites of temperature control volume boundaries
	# k_W[i,j] = conductivity of "West" boundary of [i,j] control volume

		k_W = 0.5 * (k[1:end-1, :] + k[2:end, :])
	
	# k_N[i,j] = conductivity of "North" boundary of [i,j] control volume
	
	k_N = 0.5 * (k[:, 1:end-1] + k[:, 2:end])
	
	# Initialize K matrix
	K = spzeros((m * n), (m * n))
	
	# Number control volumes based on node coordinates, going column-by-column, for m rows and n columns
	function cord2num(i, j, m)
	cv_num = i + (j - 1) * m
	return cv_num
	end
	
	# Construct K matrix
	# K[x,y] tells the heat flux from temperature volume number x to volume number y
	for i = 1:m, j = 1:(n-1)
	K[cord2num(i, j, m), cord2num(i, j + 1, m)] = -k_W[i, j+1] * (Δy / Δx)
	K[cord2num(i, j + 1, m), cord2num(i, j, m)] = -k_W[i, j+1] * (Δy / Δx)
	end
	
	for i = 1:(m-1), j = 1:n
	K[cord2num(i, j, m), cord2num(i + 1, j, m)] = -k_N[i+1, j] * (Δx / Δy)
	K[cord2num(i + 1, j, m), cord2num(i, j, m)] = -k_N[i+1, j] * (Δx / Δy)
	end
	
	# Diagonal elements of K balance out column sums
	for j = 1:(m*n)
	K[j, j] = -sum(K[:, j])
	end
	
	######################
	## Add in effect of ##
	##    Heat Sink     ##
	######################
	
	# Add heat sink in middle of left side of material by adding conductivity to diagonal element of K in corresponding row
	if iseven(m)
	# Nearest half integer
	hm = m ÷ 2
	K[cord2num(hm, 1, m), cord2num(hm, 1, m)] += k_W[hm, 1] * (Δy / Δx)
	K[cord2num(hm + 1, 1, m), cord2num(hm + 1, 1, m)] += k_W[hm+1, 1] * (Δy / Δx)
	else
	hm = m ÷ 2 + 1
	K[cord2num(hm, 1, m), cord2num(hm, 1, m)] += k_W[hm, 1] * (Δy / Δx)
	end
	
	#######################
	## Assemble Q Matrix ##
	#######################
	
	# Input vector of Heat-Generation rates
	Q = Δx*Δy*q*ones(m, n)

	######################
	## Compute T Vector ##
	######################
	
	# Solve KT = Q
	T = K \ vec(Q)
	
	###########################
	## Gradient Computations ##
	###########################
	
	if length(grad) > 0

		grad = reshape(grad, m + 1, n + 1)
		
		############################
		##    Compute λ vector    ##
		## (Dual Vector for f_av) ##
		############################
		
		λ = K \ (-ones((m * n), 1) * (1 / (m * n)))
		
		#########################
		## Create ∂k/∂η Matrix ##
		#########################
		
		dk = (p * (k₊ - k₀)) .* η .^ (p - 1)
		
		###########################
		## Assemble ∂K/∂η Matrix ##
		###########################
		
		for i = 1:m+1, j = 1:n+1
			
			###########################
			## Assemble ∂K/∂k Matrix ##
			##     for each (i,j)    ##
			###########################

			dK = spzeros((m * n), (m * n))

			if 2 <= j <= n
				for a = max(1, i - 1):min(i, m)
					dK[cord2num(a, j, m), cord2num(a, j - 1, m)] = -0.5 * (Δy / Δx)
					dK[cord2num(a, j - 1, m), cord2num(a, j, m)] = -0.5 * (Δy / Δx)
				end
			end
			if 2 <= i <= m
				for b = max(1, j - 1):min(j, n)
					dK[cord2num(i, b, m), cord2num(i - 1, b, m)] = -0.5 * (Δx / Δy)
					dK[cord2num(i - 1, b, m), cord2num(i, b, m)] = -0.5 * (Δx / Δy)
				end
			end
			for a = max(1, i - 1):min(i, m), b = max(1, j - 1):min(j, n)
				dK[cord2num(a, b, m), cord2num(a, b, m)] = -sum(dK[cord2num(a, b, m), :])
			end

			######################
			## Add in effect of ##
			##    Heat Sink     ##
			######################

			if iseven(m)
				hm = m ÷ 2
				if j == 1 && (hm ≤ i ≤ hm + 1)
					dK[cord2num(hm, 1, m), cord2num(hm, 1, m)] += 0.5 * (Δy / Δx)
				end
				if j == 1 && (hm + 1 ≤ i ≤ hm + 2)
					dK[cord2num(hm + 1, 1, m), cord2num(hm + 1, 1, m)] += 0.5 * (Δy / Δx)
				end
			else
				hm = m ÷ 2 + 1
				if j == 1 && (hm ≤ i ≤ hm + 1)
					dK[cord2num(hm, 1, m), cord2num(hm, 1, m)] += 0.5 * (Δy / Δx)
				end
			end

			###########################
			## Find Nonzero elements ##
			##  of ∂K/∂η_{i,j} and   ##
			## Assemble ∂f_av Matrix ##
			###########################
			grad[i, j] = 0.0
			A, B, Va = findnz(dK)
			for k = 1:nnz(dK)
				a = A[k]
				b = B[k]
				v = Va[k]
				grad[i, j] += λ[a] * v * T[b]
			end
			
			#############################
			## (∂K/∂k)*(∂k/∂η) = ∂K/∂η ##
			#############################
			
			grad[i, j] *= dk[i, j]
		end
	end

	##########################
	## Compute f(T) = T_avg ##
	##########################
	
	f_avg = sum(T) / (m * n)
	
	########################
	## Debugging Messages ##
	########################
	
	#println("fav = $f_avg\n", "grad = $grad\n") #Output for Debugging Purposes
	
	return f_avg
end

##################################
## Porosity Constraint Function ##
##################################

function por(x::Vector, grad::Vector, m, n)
	if length(grad) > 0
		grad .= 1.0
	end
	con = sum(x) - 0.1 * (m + 1) * (n + 1)
	#println("con = $con\n", "grad = $grad\n") #Output for Debugging Purposes
	return con
end

###################################
## Add Objective and Constraints ## 
##       to opt structure        ##
###################################

min_objective!(opt, (x, g) -> av_temp(x, g, p, m, n))

inequality_constraint!(opt, (x, g) -> por(x, g, m, n), 1e-8)

opt.lower_bounds = 0
opt.upper_bounds = 1

opt.xtol_rel = εᵢ

η = 0.05 .* ones((m + 1) * (n + 1))

f_0 = 10.0 * av_temp(η, [], p, m, n)

p_vec = []
iter_vec = []
f_av_vec = []
total_iterations = 0
total_iter_vec = []

while true
	(minf, minx, ret) = optimize!(opt, η)
	numevals = opt.numevals # the number of function evaluations
	println("$p: $minf for $numevals iterations (returned $ret)")
	global total_iterations += numevals
	global total_iter_vec = push!(total_iter_vec, total_iterations)
	global p_vec = push!(p_vec, p)
	global f_av_vec = push!(f_av_vec, minf)
	global iter_vec = push!(iter_vec, numevals)
	global p += p₊
	err = norm(minf - f_0)
	global f_0 = minf
	((err <= ε₀) || (p > p_max)) && break
end

η = reshape(η, m + 1, n + 1)

η_map = heatmap(
	0:Δx:xlen,
	0:Δy:ylen,
	η,
	yflip = true,
	xmirror = true,
	aspect_ratio = :equal,
	fontfamily = "serif",
	font = "Computer Modern Roman",
	colorbar_title = "η",
	title = "η for each Design Volume",
)
\end{jllisting}