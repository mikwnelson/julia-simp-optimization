The Volume-to-Point (VP) heat conduction problem seeks to find the optimal placement of high-conductivity material to minimize a property of the design, such as average temperature. In order to understand this problem and its solutions, we needed to study the partial differential heat equation \eqref{eqn:HeatEq} and its discretization via the Finite Volume Method. Furthermore, knowledge of optimization problems and the optimization algorithm of the Method of Moving Asymptotes was needed to implement the Solid Isotropic Material with Penalization (SIMP) algorithm to find solutions to the VP problem.
We managed to write a working implementation of the SIMP method in the Julia language that can be easily modified to different sized rectangular design domains and is able to be run on a simple desktop computer.
Areas of future investigation include implementing the harmonic average filtering scheme in \eqref{eqn:k_W-Average-Filter} and \eqref{eqn:k_N-Average-Filter} into the program. Additionally, we began investigations concerning the use of the SIMP method to maximize the eigenvalue band-gap in acoustic materials, but there was not enough time to produce results. I am also interested in investigating the use of the SIMP method for designing efficient packaging materials.