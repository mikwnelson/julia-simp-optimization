\section{Solid Isotropic Material with Penalization (SIMP)}

\subsection*{Volume-to-Point (VP) Heat Conduction Problem}

Consider a finite-size volume in which heat is being generated at \textit{every} point, and which is cooled through a small patch (the heat sink) located on its boundary. Suppose that we have a finite amount of high-conductivity ($k_+$) material available. Our goal is to determine the optimal distribution of the $k_+$ material through the given volume such that \textit{the average temperature is minimized}.

Solid Isotropic Material with Penalization (SIMP) is a method based on topology optimization that can be used to solve the VP Heat Conduction Problem. In each step of the SIMP method we increase or decrease the proportion of high-conductivity material by a small quantity. This allows us to apply methods designed for continuous optimization problems to the discrete VP problem as it transforms the binary 1---0 problem into a sequence of continuous problems \cite{Marck2012}.

\subsection{Preliminary Parameters}

\subsubsection*{Assumptions}
In order to develop the method, we need to make a couple of assumptions.

First of all, the energy differential equation driving the heat-flux inside the finite-volume requires:
\begin{enumerate}
	\item All calculations are run under steady-state conditions. That is, we seek a stable solution where quantities are independent of time.
	\item All heat produced in the volume is evacuated through the heat sink.
	\item Low-conductivity materials ($k_0$) and high-conductivity materials ($k_+$) are treated as homogeneous and isotropic on their respective conductivities.
\end{enumerate}

We also set the following conditions:
\begin{itemize}
	\item Thermal conductivities depend only on the material, and therefore are constant:
	$$k_0=1 \si[per-mode=fraction]{\watt\per\square\meter\per\kelvin}\qquad\text{and}\qquad k_+=100 \si[per-mode=fraction]{\watt\per\square\meter\per\kelvin}$$
	\item All structures have a square aspect ratio with $L=H=0.1\si{\meter}$
	\item The heat-sink is located on the middle of the west side of the structure
	\item The heat-sink has Dirichlet boundary conditions: $T_S=0\si{\celsius}$
	\item All other boundaries are adiabatic (Neumann boundary conditions): $\nabla T\cdot\mathbf{n}=0$
\end{itemize}

\subsubsection*{Notation}

We have the following sets to describe the VP-problem:
\begin{itemize}
	\item $\mathbf{x}\in\Omega$ = two-dimensional spatial field
	\item[] We set $\Omega = \Omega_0\cup\Omega_+$ where
	\begin{itemize}
		\item $\Omega_0$ = portion of $\Omega$ that has conductivity $k_0$.
		\item $\Omega_+$ = portion of $\Omega$ with conductivity $k_+$. This is the portion of the space with high-conductivity material.
	\end{itemize}
\end{itemize}

\subsection{The Optimization Problem}

Using the above established notation, we develop the following optimization problem:

\begin{equation}
	\begin{tabular}{lll}
		\text{minimize }   & $f(T)$                                                   & \text{for } $\Omega_+$ \\
		\text{subject to } & $\nabla\cdot\left(k\nabla T\right)+q=0$  &                                      \\
		& $\mathbf{x}\in\Omega$ &                                      
	\end{tabular}\label{SIMP-Optimization-Problem}
\end{equation}

The objective function $f(T)$ varies depending on desired design outcomes. Some possible objective functions include average temperature (used in the implementation in this paper), temperature variance, and maximum temperature.

Additionally, we impose a constraint upon this problem to limit the quantity of $k_+$ material available.

\begin{equation}
	\int_{\Omega_+}\dif\mathbf{x}=\int_{\Omega}\delta_+\dif\mathbf{x}\leq V\quad\text{where }\begin{cases}\delta_+ = 0 & \text{if }\mathbf{x}\in\Omega_0 \\ \delta_+ = 1 & \text{if }\mathbf{x}\in\Omega_+\end{cases}\label{eqn:vol-constraint}
\end{equation}

Equation \eqref{eqn:vol-constraint} sets a cap on the maximum volume ($V$) of $\Omega_+$ and hence limits the available amount of high-conductivity material that can be applied to the domain. If we did not have this constraint, the optimal solution would be to make $\Omega_+=\Omega$, making the entire domain have high conductivity material. However, this is not an interesting or realistic problem and applying the new constraint turns the problem into optimizing the distribution of $\Omega_0$ and $\Omega_+$ regions to minimize the chosen objective function.

\subsubsection*{Penalization Process}
The problem of whether to place high conductivity material in a particular location or not is discrete in nature. This is unfortunate as continuous optimization problems are generally easier to solve. In particular, we cannot apply some of the optimization methods described earlier, such as gradient descent, to a discrete optimization problem as they require the optimization variables to be continuous.

The SIMP method has a clever way of getting around this particular issue of discrete variables: create a continuous function that allows for a ``mix'' of the two conductive materials. This function turns our discrete variables into a continuous one, allowing us to apply the nice methods used in continuous optimization problems. However, in reality, we cannot actually mix the two conductive materials and therefore need a solution that produces a 1---0 structure. That is, our final result needs to either have conductivity $k_0$ or $k_+$ at each point, not some fraction of each. Therefore, in each iteration of the SIMP process, we \textit{penalize} the mixing of the material. Keeping this in mind we introduce a design parameter $\eta\in\left[0,1\right]$ that controls the amount of mixing of the two materials:
\begin{equation}
	k\left(\eta\right)=k_0+\left(k_+-k_0\right)\eta^p\qquad\text{with}\qquad 0\leq\eta\leq 1\qquad\text{and}\qquad p\geq1.\label{eqn:penalization}
\end{equation}
An added bonus of this formulation of $k(\eta)$ is that it is of the form \eqref{eqn:Convexity}, and hence a convex function! Notice that when $\eta=0$, $k\left(0\right)=k_0$ and when $\eta=1$, $k\left(1\right)=k_+$. The value $p$ in \eqref{eqn:penalization} is the \textit{penalization parameter}. $p$ aids in the convergence process; without $p$ the SIMP method converges to a structure that is not 1---0: a composite structure where finite-volumes are made up of different proportions of $k_0$ and $k_+$ materials.

To converge to a binary 1---0 structure, we gradually increase $p$ beginning from $p=1$. Increasing $p$ from $1$ puts the objective function in \eqref{SIMP-Optimization-Problem} at a disadvantage if $\eta\neq1$. Once $p$ gets much larger than $1$, the second term in $k(\eta)$ of \eqref{eqn:penalization} becomes much smaller than $k_0$ and hence $k(\eta)\approx k_0$ for values of $\eta\neq1$. As a result, when trying to optimize $f(T)$, values of $\eta$ in $(0,1)$ are penalized which leads to design parameters taking on values of $0$ or $1$, creating a 1---0 structure.

We can think of increasing $p$ as increasing the cost of adding $k_+$ material to the design. The volume constraint \eqref{eqn:vol-constraint} caps the amount of $k_+$ material we may add, and as $p$ increases, the optimizer needs to choose between placing ever more expensive $k_+$ material or opting for $k_0$ material, which caries with it no cost. Therefore, as the process continues, conductivity of the control volumes is either firmly $k_+$ (1 in the 1---0 terminology) or decreased to $k_0$ (the 0 in 1---0) and intermediate values are phased out, producing a binary design.

\subsection{Finite Volume Method Discretization}

Any discretization method could be used to numerically solve the heat equation \eqref{eqn:HeatEq}. In our implementation of the SIMP algorithm, the Finite Volume Method (described earlier in \S\ref{sec:FVM}) was used. FVM is used to discretize the formation of the heat equation in \eqref{SIMP-Optimization-Problem}:
\begin{equation}
	\nabla\cdot\left(k\nabla T\right)+q=0\label{eqn:SIMP-Heat-Eq}.
\end{equation}
We create a rectangular grid of $N_T=m\times n$ temperature control volumes of size $\Delta x\times\Delta y$ (solid squares in Figure \ref{fig:grids}). Each element is indexed by $1\leq i\leq m$ and $1\leq j\leq n$, where $i$ refers to the row and $j$ the column of the control volume. The volume indexed $(i,j)=(1,1)$ is located in the upper-left and $(i,j)=(m,n)$ in the bottom-right corner of the object. The temperature over the area of the temperature control volume $(i,j)$ is considered to be $T^{i,j}$.

Around the upper left corner of each temperature control volume we create a corresponding design element (dashed squares in Figure \ref{fig:grids}). (Staggering the control volume and design grids helps avoid checkerboard solutions, discussed in $\S$\ref{sec:checkerboarding}.) The area within each $(i,j)$ design element has conductivity $k^{i,j}=k(\eta^{i,j})$ as evaluated by \eqref{eqn:penalization}.

\begin{figure}
	\centering
	\ctikzfig{Chapter_II_SIMP_Optimization/Images/Grid}
	\caption[Overlayed Design \& Temperature Grids]{Overlayed Temperature (\rule[0.55ex]{1.25em}{1.5pt}) and Design (\rule[0.6ex]{0.4em}{0.5pt}\,\rule[0.6ex]{0.4em}{0.5pt}\,\rule[0.6ex]{0.4em}{0.5pt}) grids with $4\times4$ Design Element $(\eta^{i,j})$ and $3\times3$ Temperature Control Volume $(K_C^{i,j})$ Nodes. $K_N^{i,j}$ and $K_W^{i,j}$ indicate nodes at the North and West boundaries, respectively, of each Temperature Control Volume. Each Temperature control volume is numbered beginning in the upper left and continuing column-by-column, left-to-right.}
	\label{fig:grids}
\end{figure}

In this staggered grid scheme, one of the grids contains the information related to the temperature scalars and the other stores information related to the design parameters, $\eta$.

In order to employ the finite volume method, it is necessary to be able to calculate the temperature fluxes along the boundaries of each control volume. To do this, we need to have a value for the conductivity along the faces of each control volume. Notice that the faces of the temperature control volumes lie within two adjacent design regions, which implies that there are two different conductivites along that face. To create a consistent conductivity along the control volume wall, we average (using either an arithmetic or harmonic mean) the conductivity of the two adjacent design nodes. Hence, the conductivity along the West face of control volume $(i,j)$, denoted by $k^{i,j}_W$, is given by
\begin{equation}
	\text{Arithmetic Mean: }k^{i,j}_W=\frac{k^{i,j}+k^{i+1,j}}{2}\qquad\text{or}\qquad\text{Harmonic Mean: }k^{i,j}_W=2\left(\frac{1}{k^{i,j}}+\frac{1}{k^{i+1,j}}\right)^{-1}.\label{eqn:k_W-Average-Filter}
\end{equation}
Similarly, the conductivity along the North face of control volume $(i,j)$, denoted by $k^{i,j}_N$, is given by
\begin{equation}
	\text{Arithmetic Mean: }k^{i,j}_N=\frac{k^{i,j}+k^{i,j+1}}{2}\qquad\text{or}\qquad\text{Harmonic Mean: }k^{i,j}_N=2\left(\frac{1}{k^{i,j}}+\frac{1}{k^{i,j+1}}\right)^{-1}.\label{eqn:k_N-Average-Filter}
\end{equation}

For temperature control volume $(i,j)$, the finite volume method discretizes \eqref{eqn:SIMP-Heat-Eq} into the following linear equation
\begin{equation}
	K^{i,j}_C T^{i,j}=K_W^{i,j}T^{i,j-1}+K_W^{i,j+1}T^{i,j+1}+K_N^{i,j}T^{i-1,j}+K_N^{i+1,j}T^{i+1,j}+\Delta x\Delta y Q^{i,j},\label{eqn:FVM-Discret}
\end{equation}
where the $K^{i.j}$ terms represent the diffusive flux coefficients, $T^{i,j}$ the temperature of control volume $(i,j)$, and $Q^{i,j}$ the heat generation of volume $(i,j)$.

The value of the flux at the center node of the control volume is equal to the total flux through the volume faces, so we have an additional equation to pair with \eqref{eqn:FVM-Discret}:
\begin{equation}
	K^{i,j}_C=K_W^{i,j}+K_W^{i,j+1}+K_N^{i,j}+K_N^{i+1,j}\label{eqn:CenterFluxCoeff}
\end{equation}

The $K_W$ and $K_N$ coefficients are dependent on the thermal conductivity and cross-sectional area of their corresponding faces:
\begin{equation}
	K_W^{i,j}=\frac{k_W^{i,j}\Delta y}{\Delta x}\qquad\text{and}\qquad K_N^{i,j}=\frac{k_N^{i,j}\Delta x}{\Delta y}\label{eqn:K-Coeffs}
\end{equation}

The domain has Neumann boundary conditions everywhere except for the heat sink. This is very easy to implement with the finite volume method: there is no flux through these boundaries, so the flux coefficients are 0. To account for the heat-sink in the middle of the left boundary of the domain, which has Dirichlet boundary conditions, we add a ``ghost cell'' to the other side of the boundary that has temperature opposite the cell along the heat-sink in the domain. Adding this ghost cell in this way averages out the temperature on the boundary to zero.

Putting together \eqref{eqn:k_W-Average-Filter}, \eqref{eqn:k_N-Average-Filter}, \eqref{eqn:FVM-Discret}, \eqref{eqn:CenterFluxCoeff}, \eqref{eqn:K-Coeffs}, and considering boundary conditions for all $N_T$ control volumes gives us a system of equations that discretize \eqref{eqn:SIMP-Heat-Eq}. Collecting the coefficients $K$ into a matrix, representing the $T$ and $Q$ values as vectors, and doing a little reorganizing, we can represent the system of equations as a matrix equation:
\begin{equation}
	\mathbf{K}\mathbf{T}=\Delta x\Delta y\mathbf{Q}.\label{eqn:KTQ-Matrix-Eqn}
\end{equation}

Let us take a moment to analyze the structure of the matrix $\mathbf{K}$, as it might not be immediately evident to the reader what the elements of this matrix represent. (It took the author some time to interpret the meaning of this matrix.) $\mathbf{K}$ is a sparse, symmetric, and pentadiagonal $mn\times mn$ matrix.

The entries in the matrix $\mathbf{K}$ indicate the coefficient of diffusive flux between numbered temperature control volumes. Notice in Figure \ref{fig:grids} how the temperature control volumes are numbered down the columns. We can convert between volume index $(i,j)$ and control volume number, $\#$, using a simple function
\begin{equation}
	\#(i,j)=i+m(j-1)\label{eqn:cord2num}
\end{equation}

Entry $\mathbf{K}[\alpha,\beta]$ is the flux coefficient between volumes number $\alpha$ and $\beta$. Since the flux coefficient between volumes $\alpha$ and $\beta$ is the same as that between $\beta$ and $\alpha$, $\mathbf{K}[\alpha,\beta]=\mathbf{K}[\beta,\alpha]$, producing the symmetry of matrix $\mathbf{K}$. Since a particular control volume only interfaces with adjacent cells (and itself), each row/column will have (up to) five non-zero entries (all other entries are zero since there is no flux between cells that are not in contact with one another), which produces the pentadiagonality and sparsity of $\mathbf{K}$. The $i$\textsuperscript{th} elements of the size $mn$ vectors $\mathbf{T}$ and $\mathbf{Q}$ are the values of the temperature and heat generation of control volume number $i$.

Equation \eqref{eqn:KTQ-Matrix-Eqn} is solved for $\mathbf{T}$ using any appropriate method, such as LU factorization. In our implementation we used the standard ``$\backslash$'' operator in Julia: $\mathbf{T}=\mathbf{K}\backslash\mathbf{Q}$.

\subsection{Discretized Optimization Problem}

Now that we have been able to discretize \eqref{eqn:SIMP-Heat-Eq}, we can update the optimization problem \eqref{SIMP-Optimization-Problem}:

\begin{equation}
	\begin{tabular}{ll}
		\text{minimize }   & $f(\mathbf{T})$                                                                                     \\
		\text{subject to } & $\mathbf{K}\mathbf{T}=\mathbf{Q}$                                                                   \\
		                   & $\displaystyle\frac{1}{N_T}\mathbf{1}^T\boldsymbol{\eta}\leq\overline{\phi}$ \\
		                   & $\mathbf{k}=k_0\mathbf{1}+(k_+-k_0)\boldsymbol{\eta}^p$                                       \\
		                   & $0\leq\boldsymbol{\eta}\leq 1\text{ and } p\geq 1$
	\end{tabular}\label{Discretized-SIMP-Optimization-Problem}
\end{equation}

Here $\boldsymbol{\eta}$ represents the vector of design parameter values for each control volume. Additionally, we introduce the variable $\overline{\phi}$ which represents the maximum porosity, the maximal fraction of high-conductivity material allowed within the domain.

For our implementation of the SIMP method for this problem, the average temperature was chosen as the objective function:
\begin{equation}
	f_{av}\left(\mathbf{T}\right)=\frac{1}{N_T}\mathbf{1}^T\mathbf{T}.\label{eqn:f_av}
\end{equation}

Additionally, we opted to have constant and uniform heat generation for each control volume. Specifically, in our implementation we set the heat generation for each volume to be 1:
\begin{equation}
	\mathbf{Q}=\mathbf{1}.\label{eqn:Q_vec}
\end{equation}

We use the Method of Moving Asymptotes (MMA) to update the design parameters $\boldsymbol{\eta}$ throughout the optimization process. To create a local and convex approximation of the problem (see $\S$\ref{sec:MMA}), MMA requires both function and constraint evaluations, as well as evaluations of the respective gradients. Hence, we need to calculate the gradients of the average temperature function and porosity constraint. The gradients of the functions with respect to the design parameters indicate the \textit{sensitivity} of those functions to changes in $\boldsymbol{\eta}$, and hence analysis of these derivatives is called \textit{sensistivity analysis}.

\subsubsection*{Sensitivity Analysis}

We seek to find the partial derivatives of $f_{av}$ (the objective function) and $\phi$ (the constraint) with respect to an arbitrary design element $\eta_{\ell}$ so that we can form the gradients. That is, we are looking to find expressions for $$\displaystyle\frac{\partial f_{av}}{\partial\eta_\ell}\quad\text{and}\quad\displaystyle\frac{\partial\phi}{\partial\eta_\ell}.$$

The adjoint method is employed to make the calculation of the partial derivative of $f_{av}$ easier to find. By assuming \eqref{eqn:KTQ-Matrix-Eqn} is true, we add a clever form of $\mathbf{0}$ to the objective function $f_{av}$:
\begin{equation}
	f_{av}\left(\mathbf{T}\right)=\frac{1}{N_T}\mathbf{1}^T\mathbf{T}+\boldsymbol{\lambda}^T\underbrace{\left(\mathbf{K}\mathbf{T}-\mathbf{Q}\right)}_{=0}.\label{eqn:f_av-AdjointTrick}
\end{equation}

The new variable, $\boldsymbol{\lambda}=\left(\lambda_1,\lambda_2,\ldots,\lambda_{N_T}\right)$, is called the adjoint vector and behaves like a Lagrange multiplier \cite{Johnson2021}. Differentiating \eqref{eqn:f_av-AdjointTrick} with respect to an arbitrary design variable $\eta_\ell$ gives
\begin{equation}
	\frac{\partial f_{av}}{\partial\eta_{\ell}}=\frac{\mathbf{1}^T}{N_T}\frac{\partial\mathbf{T}_i}{\partial\eta_{\ell}}+\boldsymbol{\lambda}^T\left(\frac{\partial\mathbf{K}}{\partial\eta_{\ell}}\mathbf{T}+\mathbf{K}\frac{\partial\mathbf{T}}{\partial\eta_{\ell}}-\frac{\partial\mathbf{Q}}{\partial\eta_{\ell}}\right).\label{df_av/deta}
\end{equation}

The heat-generation rate is assumed to be homogeneous over design volumes and independent of the conductive material, so it does not depend on the design parameters: 
\begin{equation}
	\frac{\partial\mathbf{Q}}{\partial\eta_{\ell}}=0.\label{eqn:dQ=0}
\end{equation}
Factoring terms including the partial derivative of $\mathbf{T}$ and taking into account \eqref{eqn:dQ=0}, \eqref{df_av/deta} becomes
\begin{equation}
	\frac{\partial f_{av}}{\partial\eta_{\ell}}=\underbrace{\left(\boldsymbol{\lambda}^T\mathbf{K}+\frac{\mathbf{1}^T}{N_T}\right)}_{(\star)}\frac{\partial\mathbf{T}}{\partial\eta_{\ell}}+\boldsymbol{\lambda}^T\frac{\partial\mathbf{K}}{\partial\eta_{\ell}}\mathbf{T}.\label{df_av/delta-simplified}
\end{equation}
Hence $\frac{\partial\mathbf{T}}{\partial\eta_{\ell}}$ is eliminated from the expression if $\boldsymbol{\lambda}$ is taken such that $(\star)$ in \eqref{df_av/delta-simplified} equals $0$:
\begin{equation}
	\boldsymbol{\lambda}^T\mathbf{K}+\frac{\mathbf{1}^T}{N_T}=0\iff\mathbf{K}\boldsymbol{\lambda}=-\frac{\mathbf{1}}{N_T}\mathbf{1}^T.\label{eqn:lambda-vector-equation}
\end{equation}
\eqref{eqn:lambda-vector-equation} is solved the same way as \eqref{eqn:KTQ-Matrix-Eqn}. Thus, using the values for $\boldsymbol{\lambda}$ from solving \eqref{eqn:lambda-vector-equation},
\begin{equation}
	\frac{\partial f_{av}}{\partial\eta_{\ell}}=\boldsymbol{\lambda}^T\frac{\partial\mathbf{K}}{\partial\eta_{\ell}}\mathbf{T}.\label{eqn:d-f_av}
\end{equation}
$\boldsymbol{\lambda}$ and $\mathbf{T}$ are found by solving \eqref{eqn:lambda-vector-equation} and \eqref{eqn:KTQ-Matrix-Eqn}, respectively. Thus, it remains to find $\frac{\partial\mathbf{K}}{\partial\eta_{\ell}}$. Recall that matrix $\mathbf{K}$ contains the coefficients of control volume boundary conductivities, which depend on the penalization equation $\mathbf{k}(\eta)$ \eqref{eqn:penalization}. Hence, using the chain rule,
\begin{equation}
	\frac{\partial\mathbf{K}}{\partial\eta_{\ell}}=\frac{\partial\mathbf{K}}{\partial\mathbf{k}}\frac{\partial\mathbf{k}}{\partial\eta_{\ell}}.\label{eqn:chain-rule}
\end{equation}
By \eqref{eqn:penalization}
\begin{equation}
	\frac{\partial\mathbf{k}}{\partial\eta_{\ell}}=
	\begin{cases}
		p\left(k_{+}-k_0\right)\eta_{\ell}^{p-1} & \text{for the }\ell\text{th element},\\
		0 & \text{otherwise}.
	\end{cases}
\end{equation}
Using \eqref{eqn:k_W-Average-Filter}, \eqref{eqn:k_N-Average-Filter}, \eqref{eqn:CenterFluxCoeff}, and \eqref{eqn:K-Coeffs} one is able to find $\frac{\partial\mathbf{K}}{\partial\mathbf{k}}$, the form of which will vary slightly based on whether an arithmetic or harmonic mean is chosen. For the arithmetic mean
\begin{equation}
	\frac{\partial\mathbf{K}}{\partial\mathbf{k}}=
	\begin{cases}
		-\frac{1}{2}\frac{\Delta y}{\Delta x} & \text{if }k\text{ corresponds to a }k_W\text{ face},\\
		-\frac{1}{2}\frac{\Delta x}{\Delta y} & \text{if }k\text{ corresponds to a }k_N\text{ face},\\
		-\left(\frac{\Delta y}{\Delta x}+\frac{\Delta x}{\Delta y}\right) & \text{if }k\text{ corresponds to a }k_C\text{ node}.
	\end{cases}
\end{equation}
It is now possible to compute $\displaystyle\frac{\partial f_{av}}{\partial\eta_{\ell}}$ using the above. We still require $\displaystyle\frac{\partial\phi}{\partial\eta_{\ell}}$.

The porosity function is
\begin{equation}
	\phi(\boldsymbol{\eta})=\frac{1}{N_\eta}\mathbf{1}^T\boldsymbol{\eta}\label{eqn:porosity}
\end{equation}
where $N_\eta$ indicates the number of design elements. The optimizer used in the implementation (NLopt) requires constraints to be in the form of a function less than or equal to zero, so we reorganize the form of \eqref{eqn:porosity} in \eqref{Discretized-SIMP-Optimization-Problem}:
\begin{equation}
	\mathbf{1}^T\boldsymbol{\eta}-N_{T}\overline{\phi}\leq 0\label{ineq:porosity}
\end{equation}
Therefore, we set the left-hand side of this inequality to be our porosity function:
\begin{equation}
	\tilde{\phi}\left(\boldsymbol{\eta}\right)=\mathbf{1}^T\boldsymbol{\eta}-N_T\overline{\phi}\label{eqn:modified-porosity}
\end{equation}
Therefore,
\begin{equation}
	\frac{\partial\tilde{\phi}}{\partial\eta_{\ell}}=1\label{eqn:partial-por}
\end{equation}
Now, we update the optimization problem one last time with the results from \eqref{eqn:f_av}, \eqref{eqn:Q_vec}, and \eqref{ineq:porosity}.

\begin{equation}
	\begin{tabular}{ll}
		$\text{minimize}$  & $f_{av}\left({\mathbf{T}}\right)=\frac{1}{N_T}\mathbf{1}^T\mathbf{T}$                                                                                     \\
		\text{subject to } & $\mathbf{K}\mathbf{T}=\mathbf{1}$                                                                   \\
		& $\mathbf{1}^T\boldsymbol{\eta}-N_T\overline{\phi}\leq0$ \\
		& $\mathbf{k}=k_0\mathbf{1}+(k_+-k_0)\boldsymbol{\eta}^p$                                       \\
		& $0\leq\boldsymbol{\eta}\leq 1\text{ and } p\geq 1$
	\end{tabular}\label{Implementation-SIMP-Optimization-Problem}
\end{equation}

\eqref{Implementation-SIMP-Optimization-Problem} is finally in a form that we can input the problem into the NLopt optimizer using the MMA algorithm and produce possible design solutions to the VP heat conduction problem.